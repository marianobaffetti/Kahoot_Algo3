\documentclass[titlepage,a4paper]{article}

\usepackage{a4wide}
\usepackage[colorlinks=true,linkcolor=black,urlcolor=blue,bookmarksopen=true]{hyperref}
\usepackage{bookmark}
\usepackage{fancyhdr}
\usepackage[spanish]{babel}
\usepackage[utf8]{inputenc}
\usepackage[T1]{fontenc}
\usepackage{graphicx}
\usepackage{float}

\pagestyle{fancy} % Encabezado y pie de página
\fancyhf{}
\fancyhead[R]{Algoritmos y Programación III - FIUBA}
\renewcommand{\headrulewidth}{0.4pt}
\fancyfoot[C]{\thepage}
\renewcommand{\footrulewidth}{0.4pt}

\begin{document}
\begin{titlepage} % Carátula
	\hfill\includegraphics[width=6cm]{logofiuba.jpg}
    \centering
    \vfill
    \Huge \textbf{Trabajo Práctico 2 — Java}
    \vskip2cm
    \Large [7507/9502] Algoritmos y Programación III\\
    Curso 1 \\ % Curso 1 para el de la tarde y 2 para el de la noche
    Grupo 14 \\
    Primer cuatrimestre de 2020 
    \vfill
    Integrantes\\[1\baselineskip]
    
    Baffetti Mariano\\
    Cuppari Juan\\
    Hoszowski Juan\\
    Iskandarani Roberto\\
    Perez Ignacio\\
    \vfill
    Corrector: Edson Justo
    \vfill
\end{titlepage}

\tableofcontents % Índice general
\newpage

\section{Introducción}\label{sec:intro}
El presente informe reune la documentación de la solución del segundo trabajo práctico de la materia Algoritmos y Programación III que consiste en desarrollar un juego de preguntas y respuestas, denominado Kahoot, en Java utilizando los conceptos del paradigma de la orientación a objetos vistos hasta en el curso.


\section{Supuestos}\label{sec:supuestos}
\begin{itemize}
    \item
\end{itemize}


\section{Diagramas de Clases}\label{sec:diagramasdeclases}
% Mostrar las secuencias interesantes que hayan implementado. Pueden agregar texto para explicar si algo no queda claro.

\begin{figure}[H]
\centering
\includegraphics[width=1.1\textwidth]{sequence-diagram-1.png}
\caption{\label{fig:seq01}Responder una pregunta y asignar puntos.}
\end{figure}


\section{Diagramas de secuencia}\label{sec:diagramasdesecuencia}
% Mostrar las secuencias interesantes que hayan implementado. Pueden agregar texto para explicar si algo no queda claro.

\begin{figure}[H]
\centering
\includegraphics[width=1.1\textwidth]{sequence-diagram-1.png}
\caption{\label{fig:seq01}Responder una pregunta y asignar puntos.}
\end{figure}


\section{Diagramas de Paquetes}\label{sec:diagramasdepaquetes}

\begin{figure}[H]
\centering
\includegraphics[width=1.1\textwidth]{sequence-diagram-1.png}
\caption{\label{fig:seq01}Responder una pregunta y asignar puntos.}
\end{figure}


\section{Diagramas de Estado}\label{sec:diagramasdeestado}

\begin{figure}[H]
\centering
\includegraphics[width=1.1\textwidth]{sequence-diagram-1.png}
\caption{\label{fig:seq01}Responder una pregunta y asignar puntos.}
\end{figure}


\section{Detalles de Implementación}\label{sec:detallesdeimplementacion}

\subsection{}


\section{Excepciones}\label{sec:excepciones}

\begin{description}
\item[]

\end{description}


\end{document}
