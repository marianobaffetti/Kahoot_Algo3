\documentclass[titlepage,a4paper]{article}

\usepackage{a4wide}
\usepackage[colorlinks=true,linkcolor=black,urlcolor=blue,bookmarksopen=true]{hyperref}
\usepackage{bookmark}
\usepackage{fancyhdr}
\usepackage[spanish]{babel}
\usepackage[utf8]{inputenc}
\usepackage[T1]{fontenc}
\usepackage{graphicx}
\usepackage{float}

\pagestyle{fancy} % Encabezado y pie de página
\fancyhf{}
\fancyhead[R]{Algoritmos y Programación III - FIUBA}
\renewcommand{\headrulewidth}{0.4pt}
\fancyfoot[C]{\thepage}
\renewcommand{\footrulewidth}{0.4pt}

\begin{document}
\begin{titlepage} % Carátula
	\hfill\includegraphics[width=6cm]{logofiuba.jpg}
    \centering
    \vfill
    \Huge \textbf{Trabajo Práctico 2 — Java}
    \vskip2cm
    \Large [7507/9502] Algoritmos y Programación III\\
    Curso 1 \\ % Curso 1 para el de la tarde y 2 para el de la noche
    Grupo 14 \\
    Primer cuatrimestre de 2020 
    \vfill
    Integrantes\\[1\baselineskip]
    
    Baffetti Mariano\\
    Cuppari Juan\\
    Hoszowski Juan\\
    Iskandarani Roberto\\
    Perez Ignacio\\
    \vfill
    Corrector: Edson Justo
    \vfill
\end{titlepage}

\tableofcontents % Índice general
\newpage

\section{Introducción}\label{sec:intro}
El presente informe reune la documentación de la solución del segundo trabajo práctico de la materia Algoritmos y Programación III que consiste en desarrollar un juego de preguntas y respuestas, denominado Kahoot, en Java utilizando los conceptos del paradigma de la orientación a objetos vistos hasta en el curso. En el juego participan al menos dos jugadores, que deben ir respondiendo por turnos. Las preguntas pueden ser de distinto tipo (Verdadero o falso, Multiple Choice, Ordered Choice, Group Choice) y además pueden tener bonificaciones asociadas (X2, X3, exclusividad de puntaje). Al finalizar el juego se indica quien ha sido el jugador que obtuvo el mayor puntaje.


\section{Supuestos}\label{sec:supuestos}
\begin{itemize}
    \item El tiempo para cada turno es de 30 segundos.
\end{itemize}
\begin{itemize}
    \item Un jugador puede tener puntaje negativo, esto se da debido a las preguntas con penalidad ya que por cada opción incorrecta se le resta un punto.
\end{itemize}
\begin{itemize}
    \item En caso de tener el mismo puntaje, puede haber más de un ganador.
\end{itemize}
\begin{itemize}
    \item No pueden haber dos jugadores con el mismo nombre.
\end{itemize}
\begin{itemize}
    \item En las preguntas de tipo Ordered Choice y Group Choice la respuesta se considera correcta siempre que todas las opciones seleccionadas sean correctas.
\end{itemize}
\begin{itemize}
    \item En las preguntas de tipo Ordered Choice y Group Choice se suma un punto por cada opción correcta en caso de que la respuesta sea correcta.
\end{itemize}


\section{Diagramas de Clases}\label{sec:diagramasdeclases}
% Mostrar las secuencias interesantes que hayan implementado. Pueden agregar texto para explicar si algo no queda claro.

\begin{figure}[H]
\centering
\includegraphics[width=1.1\textwidth]{sequence-diagram-1.png}
\caption{\label{fig:seq01}Responder una pregunta y asignar puntos.}
\end{figure}


\section{Diagramas de secuencia}\label{sec:diagramasdesecuencia}
% Mostrar las secuencias interesantes que hayan implementado. Pueden agregar texto para explicar si algo no queda claro.

\begin{figure}[H]
\centering
\includegraphics[width=1.1\textwidth]{sequence-diagram-1.png}
\caption{\label{fig:seq01}Responder una pregunta y asignar puntos.}
\end{figure}


\section{Diagramas de Paquetes}\label{sec:diagramasdepaquetes}

\begin{figure}[H]
\centering
\includegraphics[width=1.1\textwidth]{sequence-diagram-1.png}
\caption{\label{fig:seq01}Responder una pregunta y asignar puntos.}
\end{figure}


\section{Diagramas de Estado}\label{sec:diagramasdeestado}

\begin{figure}[H]
\centering
\includegraphics[width=1.1\textwidth]{sequence-diagram-1.png}
\caption{\label{fig:seq01}Responder una pregunta y asignar puntos.}
\end{figure}


\section{Detalles de Implementación}\label{sec:detallesdeimplementacion}

\subsection{Implementación de Multiplicadores} (proxy + strategy)
\subsection{Implementación de Kahoot} (singleton)
\subsection{Implementación de MVC} (observer)
\subsection{Implementación de ExclusividadDePuntaje} (method object)
\subsection{Implementación de Vistas de Preguntas} (factory)




\section{Excepciones}\label{sec:excepciones}

\begin{description}
\item[YaHayUnMultiplicadorEnUsoError:] Excepción que se lanza cuando un jugador intenta usar más de un multiplicador en un mismo turno, mostrando un cartel de advertencia.

\end{description}

\begin{description}
\item[YaHayUnaExclusividadEnUsoError:] Excepción que se lanza cuando un jugador intenta usar más de un bonificador de exclusividad en un mismo turno, mostrando un cartel de advertencia.

\end{description}

\begin{description}
\item[YaExisteJugadorConEseNombreError:] Excepción que se lanza cuando un jugador intenta registrarse con un nombre ya utilizado, mostrando un cartel de advertencia para que modifique el nombre ingresado.

\end{description}

\begin{description}
\item[NoSePuedeUtilizarMultiplicadorError:] Excepción que se lanza cuando un jugador intenta utilizar un multiplicador en una pregunta que no permite el uso del mismo.

\end{description}

\begin{description}
\item[NoSePuedeUtilizarExclusividadError:] Excepción que se lanza cuando un jugador intenta utilizar un bonificador de exclusividad en una pregunta que no permite el uso del mismo.

\end{description}

\begin{description}
\item[NoSePuedeIniciarJuegoSiNoHayPreguntasError:] Excepción que se lanza cuando el programa no detecta ninguna pregunta cargada.

\end{description}

\begin{description}
\item[NoSePuedeIniciarJuegoSiNoHayJugadoresError:] Excepción que se lanza cuando se intenta iniciar el juego sin tener ningún jugador registrado.

\end{description}

\begin{description}
\item[NoSeEncuentraElMultiplicadorError:] Excepción que se lanza cuando un jugador intenta utilizar un multiplicador cuando ya se agotaron, mostrando un cartel de advertencia.

\end{description}

\begin{description}
\item[NoHayExclusividadesDisponiblesError:] Excepción que se lanza cuando un jugador intenta utilizar un bonificador de exclusividad cuando ya se agotaron, mostrando un cartel de advertencia.

\end{description}

\begin{description}
\item[JugadorNoSePuedeCrearConNombreVacioError:] Excepción que se lanza cuando se intenta registrar un jugador sin haber ingresado ningún caracter, mostrando un cartel de advertencia.

\end{description}


\end{document}
