\documentclass[titlepage,a4paper]{article}

\usepackage{a4wide}
\usepackage[colorlinks=true,linkcolor=black,urlcolor=blue,bookmarksopen=true]{hyperref}
\usepackage{bookmark}
\usepackage{fancyhdr}
\usepackage[spanish]{babel}
\usepackage[utf8]{inputenc}
\usepackage[T1]{fontenc}
\usepackage{graphicx}
\usepackage{float}

\pagestyle{fancy} % Encabezado y pie de página
\fancyhf{}
\fancyhead[R]{Algoritmos y Programación III - FIUBA}
\renewcommand{\headrulewidth}{0.4pt}
\fancyfoot[C]{\thepage}
\renewcommand{\footrulewidth}{0.4pt}

\begin{document}
\begin{titlepage} % Carátula
	\hfill\includegraphics[width=6cm]{logofiuba.jpg}
    \centering
    \vfill
    \Huge \textbf{Trabajo Práctico 2 — Java}
    \vskip2cm
    \Large [7507/9502] Algoritmos y Programación III\\
    Curso 1 \\ % Curso 1 para el de la tarde y 2 para el de la noche
    Grupo 14 \\
    Primer cuatrimestre de 2020 
    \vfill
    Integrantes\\[1\baselineskip]
    
    Baffetti Mariano\\
    Cuppari Juan\\
    Hoszowski Juan\\
    Iskandarani Roberto\\
    Perez Ignacio\\
    \vfill
    Corrector: Edson Justo
    \vfill
\end{titlepage}

\tableofcontents % Índice general
\newpage

\section{Introducción}\label{sec:intro}
El presente informe reune la documentación de la solución del segundo trabajo práctico de la materia Algoritmos y Programación III que consiste en desarrollar un juego de preguntas y respuestas, denominado Kahoot, en Java utilizando los conceptos del paradigma de la orientación a objetos vistos hasta en el curso.

%\section{Supuestos}\label{sec:supuestos}
 %Deberá contener explicaciones de cada uno de los supuestos que el alumno haya tenido que adoptar a partir de situaciones que no estén contempladas en la especificación.


% \section{Modelo de dominio}\label{sec:modelo}
% Explicación concisa del diseño general del trabajo.

\section{Diagramas de clase}\label{sec:diagramasdeclase}

\begin{figure}[H]
\centering
\includegraphics[width=1.1\textwidth]{class-diagram-1.png}
\caption{\label{fig:class01}Diagrama de Ronda.}
\end{figure}

Se muestra las relaciones presentes entre las clases que resuelven la secuencia de preguntas y respuestas durante una ronda del juego.
%\section{Detalles de implementación}\label{sec:implementacion}
% Explicaciones sobre la implementación interna de algunas clases que consideren que puedan llegar a resultar interesantes.

%\subsection{Aliquam vel eros id magna vestibulum rhoncus}

%\ Código
%\\begin{verbatim}
%\| rango |
%\rango := (2 to: 20) asOrderedCollection.
%\Transcript show: rango ; cr.
%\rango copy do: [ :unNumero | unNumero isPrime ifFalse: [ rango remove: unNumero ] ].
%\Transcript show: rango.
%\\end{verbatim}

%\section{Excepciones}\label{sec:excepciones}
% Explicación de cada una de las excepciones creadas y con qué fin fueron creadas.

%\begin{description}
%\item[Exception] Lorem ipsum dolor sit amet, consectetur adipiscing elit. Proin nec facilisis odio. Pellentesque habitant morbi tristique senectus et netus et malesuada fames ac turpis egestas. In aliquam dapibus lacus at condimentum. Curabitur ornare scelerisque euismod. Duis a mi in nulla sodales sollicitudin vehicula sit amet sapien. Quisque vel eros ut libero consequat scelerisque. Nullam efficitur ante eu massa gravida sollicitudin.
%\item[Excepcion] Curabitur elementum laoreet molestie. Ut hendrerit, quam lobortis porttitor cursus, ex sem facilisis massa, in interdum odio risus hendrerit dui.
%\item[Excepcion] Integer porta efficitur felis. Etiam facilisis consectetur sem, ac efficitur orci. Nam a ante commodo, fringilla nisl a, sollicitudin est.
%\item[Excepcion] Aliquam erat volutpat. Fusce quis efficitur augue. Fusce egestas mauris a nisi finibus volutpat. Maecenas venenatis ligula ut nisi maximus, vel ultricies enim scelerisque.
%\item[Excepcion] Mauris gatis feugiat erat non euismod. Donec sagittis orci enim, et convallis lacus sodales at. Nunc laoreet leo vel metus eleifend, vel aliquam sem tincidunt. Nunc imperdiet eget erat eget tincidunt. Morbi tempus risus quis nulla faucibus facilisis. Sed varius nunc vel neque rutrum vestibulum.
%\end{description}

\section{Diagramas de secuencia}\label{sec:diagramasdesecuencia}
% Mostrar las secuencias interesantes que hayan implementado. Pueden agregar texto para explicar si algo no queda claro.



\begin{figure}[H]
\centering
\includegraphics[width=1.1\textwidth]{sequence-diagram-1.png}
\caption{\label{fig:seq01}Responder una pregunta y asignar puntos.}
\end{figure}


%\begin{figure}[H]
%\centering
%\includegraphics[width=\textwidth]{diagrama_secuencia02.png}
%\caption{\label{fig:seq02}Nam a nulla non mauris ullamcorper.}
%\end{figure}

\end{document}
